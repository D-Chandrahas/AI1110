
\documentclass[journal,12pt,twocolumn]{IEEEtran}

\usepackage{amsfonts,amsmath,amssymb}

\newcommand{\myvec}[1]{\ensuremath{\begin{pmatrix}#1\end{pmatrix}}}
\let\vec\mathbf

\begin{document}

\title{\textbf{AI1110 Assignment 1} }
\author{\textbf{Dondapati Chandrahas Reddy}\\ \textbf{AI21BTECH11010}\\ ICSE Grade 10 2019 paper}

\maketitle

{\section {Question 3 (a) \newline}}

{\large \underline{Question}:\newline}

Simplify

\begin{equation}
	\sin A\myvec{\sin A &  -\cos A \\ \cos A & \sin A} + \cos A \myvec{\cos A &  \sin A \\ -\sin A & \cos A}
\end{equation}\\

{\large \underline{Solution}:}\\

Let,
\begin{align}
	\vec{R}(\theta) &= \myvec{\cos \theta &  -\sin \theta \\ \sin \theta & \cos \theta} \\[1em] \vec{I} &= \myvec{1 & 0 \\ 0 & 1} \\[1em] \vec{J} &= \myvec{0 & -1 \\ 1 & 0}
\end{align}\\

The matrix expression in the question can be written as

\begin{equation}
	\myvec{\sin A & \cos A} \left( \vec{R}(A)^T \myvec{\vec{J} \\ \vec{I}} \right)
\end{equation}\\

Taking dot product of the vectors

\begin{equation}
	\myvec{\sin^2 A &  -\sin A\cos A \\ \sin A\cos A & \sin^2 A} +\myvec{\cos^2 A & \cos A\sin A \\ -\cos A\sin A & \cos^2 A}
\end{equation}\\

Adding the matrices

\begin{equation}
	\myvec{\sin^2 A + \cos^2 A &  -\sin A \cos A +\cos A \sin A \\ \sin A \cos A -\cos A \sin A & \sin^2 A + \cos^2 A}
\end{equation}\\

Simplifying the elements finally gives

\begin{equation}
	\myvec{1 & 0 \\ 0 & 1}
\end{equation}\\

\end{document}