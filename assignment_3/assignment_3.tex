
\documentclass[journal,12pt,twocolumn]{IEEEtran}
\usepackage{amsfonts}
\usepackage{amsmath}
\usepackage{amssymb}

\begin{document}
\title{\textbf{AI1110 Assignment 3} }
\author{\textbf{Dondapati Chandrahas Reddy}\\\textbf{AI21BTECH11010}\\ CBSE 9th Probability}
\maketitle

{\section{Example 4}}

{\large \underline{Question}:}\\

On one page of a telephone directory, there were 200 telephone numbers.
The frequency distribution of their unit place digit (for example, in the number 25828573,
the unit place digit is 3) is given in the table below.

\begin{table}[h!]
	\begin{tabular}{|c|c|c|c|c|c|c|c|c|c|c|}
		\hline
		Digit & 0 & 1 & 2 & 3 & 4 & 5 & 6 & 7 & 8 & 9 \\
		\hline
		Frequency & 22 & 26 & 22 & 22 & 20 & 10 & 14 & 28 & 16 & 20 \\
		\hline
	\end{tabular}
\end{table}

Without looking at the page, the pencil is placed on one of these numbers, i.e., the
number is chosen at random. What is the probability that the digit in its unit place is 6? \\

{\large \underline{Solution}:}\\

The probability of random number having 6 in unit's place
\begin{align}
	& = \dfrac{\text{number of phone numbers with 6 in unit's place}}{\text{total number of phone numbers}} \\
	& = \dfrac{14}{200} \\
	& = 0.07 \text{ or } 7\,\% 	
\end{align} 





\end{document}